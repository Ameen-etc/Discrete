%\iffalse
\let\negmedspace\undefined
\let\negthickspace\undefined
\documentclass[journal,12pt,twocolumn]{IEEEtran}
\usepackage{cite}
\usepackage{amsmath,amssymb,amsfonts,amsthm}
\usepackage{algorithmic}
\usepackage{graphicx}
\usepackage{textcomp}
\usepackage{xcolor}
\usepackage{txfonts}
\usepackage{listings}
\usepackage{enumitem}
\usepackage{mathtools}
\usepackage{gensymb}
\usepackage{comment}
\usepackage[breaklinks=true]{hyperref}
\usepackage{tkz-euclide} 
\usepackage{listings}
\usepackage{gvv}                                        
\def\inputGnumericTable{}                                 
\usepackage[latin1]{inputenc}                                
\usepackage{color}                                            
\usepackage{array}                                            
\usepackage{longtable}                                       
\usepackage{calc}                                             
\usepackage{multirow}                                         
\usepackage{hhline}                                           
\usepackage{ifthen}                                           
\usepackage{lscape}
\newtheorem{theorem}{Theorem}[section]
\newtheorem{problem}{Problem}
\newtheorem{proposition}{Proposition}[section]
\newtheorem{lemma}{Lemma}[section]
\newtheorem{corollary}[theorem]{Corollary}
\newtheorem{example}{Example}[section]
\newtheorem{definition}[problem]{Definition}
\newcommand{\BEQA}{\begin{eqnarray}}
\newcommand{\EEQA}{\end{eqnarray}}
\newcommand{\define}{\stackrel{\triangle}{=}}
\theoremstyle{remark}
\newtheorem{rem}{Remark}
\begin{document}

\bibliographystyle{IEEEtran}
\vspace{3cm}

\title{DISCRETE}
\author{EE23BTECH11006 - Ameen Aazam$^{*}$% <-this % stops a space
}
\maketitle
\newpage
\bigskip

\renewcommand{\thefigure}{\theenumi}
\renewcommand{\thetable}{\theenumi}

\vspace{3cm}
\textbf{Question :}
Find the sum of the following APs:
\begin{enumerate}[label=(\alph*)]
\item $2, 7, 12, \ldots$ to $10$ terms.
\item $-37, -33, -29, \ldots$ to $12$ terms.
\item $0.6, 1.7, 2.8, \ldots$ to $100$ terms.
\item $\frac{1}{15}, \frac{1}{12}, \frac{1}{10}, \ldots$ to $11$ terms.
\end{enumerate}
\solution
%\fi
\input{tables/parameters}
From \eqref{eq:abc}, we get the sum to $n$ terms,
\begin{align}
    y(n)=\frac{(n+1)}{2}\cbrak{2x(0)+nd}u(n)
\end{align}
\begin{enumerate}[label=(\alph*)]
    \item \begin{align}
        &x(0)=2 \\
        &d=5 \\
        \implies &s(9)=245
    \end{align}
    \begin{figure}[h!]
        \centering
        \includegraphics[width=0.9\columnwidth]{figs/plt1.png}
        \caption{$1st$ AP}
    \end{figure}
    \item \begin{align}
        &x(0)=-37 \\
        &d=4 \\
        \implies &s(11)=-180
    \end{align}
    \begin{figure}[h!]
        \centering
        \includegraphics[width=0.9\columnwidth]{figs/plt2.png}
        \caption{$2nd$ AP}
    \end{figure}
    \item \begin{align}
        &x(0)=0.6 \\
        &d=1.1 \\
        \implies &s(99)=5505
    \end{align}
    \begin{figure}[h!]
        \centering
        \includegraphics[width=0.9\columnwidth]{figs/plt3.png}
        \caption{$3rd$ AP}
    \end{figure}
    \item \begin{align}
        &x(0)=\frac{1}{15} \\
        &d=\frac{1}{60} \\
        \implies &s(10)=1.65
    \end{align}
    \begin{figure}[h!]
        \centering
        \includegraphics[width=0.9\columnwidth]{figs/plt4.png}
        \caption{$4th$ AP}
    \end{figure}
\end{enumerate}

\end{document}

